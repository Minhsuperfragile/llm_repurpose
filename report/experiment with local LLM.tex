\documentclass[12pt]{article}
\usepackage[a4paper,margin=1in]{geometry}
\usepackage{titlesec}
\usepackage{hyperref}
\hypersetup{
    colorlinks=true,     % Enable colored links
    linkcolor=blue,      % Color for internal links
    citecolor=blue,      % Color for citations
    filecolor=blue,      % Color for file links
    urlcolor=blue        % Color for URLs
}
\usepackage{indentfirst} % Indent the first paragraph after sections
\setlength{\parindent}{2em} % Adjust the length (e.g., 2em)

\usepackage[utf8]{inputenc} % Hỗ trợ mã hóa UTF-8
\usepackage[vietnamese]{babel} % Hỗ trợ tiếng Việt
\usepackage{vntex} % Gói hỗ trợ đặc biệt cho tiếng Việt
% Configure page layout
\geometry{a4paper, margin=1in}

% Title formatting
\titleformat{\section}[block]{\large\bfseries}{\thesection}{1em}{}

\title{Successfully Utilized Local LLM to Generate Podcast Script}
\author{}
\date{}

\begin{document}

\maketitle
\section{Methodology}
In the previous experiment with Llama3.3 on local computer, we were unable to create a script for an useful podcast. Instead, it provided a summary of the given document. 
So I took a step back, broke our problem into 3 steps to give to Llama3.3: Summarize the document, generate a set of quenstions with corresponding answers, 
generate a complete script. For each step, it takes output of previous step as input. At the end, it can finally create a useful, detailed podcast script.
From my view, I have to say that script created by our local LLM is even better than script generated by Gemini on cloud. 

This time, I used both 8-bit and 4-bit version of Llama3.3 70B. And the result of Llama3.3-Q8 is clearly better since it gave a better welcome scene, closing lines, 
and somewhat "soul" to the conversation. Both are given the same system prompt, you can take a look at \href{https://github.com/Minhsuperfragile/llm_repurpose/blob/main/podcast_creation/default_prompts.py}{this file on github}.
To the matter of size, Llama3.3-Q8 is 75Gb while Llama3.3-Q4 is only 43Gb. Inference time is also longer for Llama3.3-Q8, around 5-6 hours on CPU.

\section{Script Generated by Llama3.3 Q8}
Below is the script generated by Llama3.3 Quantized 8-bit. I used the same \href{https://thuvienphapluat.vn/van-ban/Bo-may-hanh-chinh/Quyet-dinh-17-2021-QD-UBND-vi-tri-chuc-nang-Chi-cuc-Trong-trot-va-Bao-ve-thuc-vat-tinh-Lao-Cai-481420.aspx}{vietnamese law document} as previous experiment with Gemini.
This is cut down to final output only, see full 3 steps output \href{google.com}{here}. Linh is host, and Minh is guess, these names can be changed to our desire.
%TODO: replace google.com with real github link

\noindent\fbox{%
\begin{minipage}{\textwidth}
\textcolor{blue}{\textbf{\textit{"Linh"}}}: "Xin chào mọi người, hôm nay chúng ta có sự tham gia của anh Minh, một chuyên gia trong lĩnh vực nông nghiệp. Anh Minh sẽ giúp chúng ta hiểu rõ hơn về tổ chức và hoạt động của Chi cục Trồng trọt và Bảo vệ thực vật.",\\
\textcolor{red}{\textbf{\textit{"Minh"}}}: "Cảm ơn Linh đã mời tôi. Tôi rất sẵn lòng chia sẻ thông tin về đơn vị của mình.",\\
\textcolor{blue}{\textbf{\textit{"Linh"}}}: "Chúng ta bắt đầu với câu hỏi đầu tiên: Chức năng chính của Chi cục Trồng trọt và Bảo vệ thực vật thuộc Sở Nông nghiệp và Phát triển nông thôn là gì?",\\
\textcolor{red}{\textbf{\textit{"Minh"}}}: "Chức năng chính của chúng tôi là giúp Sở Nông nghiệp và Phát triển nông thôn quản lý nhà nước về trồng trọt, bảo vệ thực vật, kiểm dịch thực vật, quản lý thuốc bảo vệ thực vật và giống cây trồng.",\\
\textcolor{blue}{\textbf{\textit{"Linh"}}}: "Được rồi, điều đó rất quan trọng. Tiếp theo, anh có thể phân tích vai trò của công tác kiểm dịch thực vật trong hoạt động của Chi cục không?",\\
\textcolor{red}{\textbf{\textit{"Minh"}}}: "Kiểm dịch thực vật là một phần quan trọng trong quản lý nhà nước về bảo vệ thực vật, đảm bảo không cho các loài gây hại xâm nhập và lan truyền trên địa bàn tỉnh.",\\
\textcolor{blue}{\textbf{\textit{"Linh"}}}: "Vậy tầm quan trọng của việc tổ chức thực hiện các dự án, chương trình về trồng trọt, bảo vệ thực vật đối với sự phát triển nông nghiệp bền vững là gì?",\\
\textcolor{red}{\textbf{\textit{"Minh"}}}: "Tổ chức thực hiện các dự án và chương trình này giúp nâng cao hiệu quả sản xuất, áp dụng công nghệ mới, và bảo vệ môi trường, từ đó góp phần vào sự phát triển nông nghiệp bền vững.",\\
\textcolor{blue}{\textbf{\textit{"Linh"}}}: "Nếu một cá nhân muốn tìm hiểu về quy định quản lý thuốc bảo vệ thực vật, họ nên liên hệ với phòng ban nào của Chi cục?",\\
\textcolor{red}{\textbf{\textit{"Minh"}}}: "Họ nên liên hệ với Phòng Bảo vệ và Kiểm dịch thực vật.",\\
\textcolor{blue}{\textbf{\textit{"Linh"}}}: "Ai có trách nhiệm ban hành quyết định về chức năng, nhiệm vụ, quyền hạn của các phòng thuộc Chi cục Trồng trọt và Bảo vệ thực vật?",\\
\textcolor{red}{\textbf{\textit{"Minh"}}}: "Chi cục trưởng Chi cục Trồng trọt và Bảo vệ thực vật.",\\
\textcolor{blue}{\textbf{\textit{"Linh"}}}: "Cơ cấu tổ chức của Chi cục bao gồm những thành phần nào?",\\
\textcolor{red}{\textbf{\textit{"Minh"}}}: "Gồm 1 Chi cục trưởng, 1 Phó Chi cục trưởng, và các phòng chuyên môn như Phòng Hành chính tổng hợp, Phòng Trồng trọt, Phòng Bảo vệ và Kiểm dịch thực vật.",\\
\textcolor{blue}{\textbf{\textit{"Linh"}}}: "Vai trò của biên chế trong hoạt động của Chi cục và cơ sở để xác định số lượng biên chế là gì?",\\
\textcolor{red}{\textbf{\textit{"Minh"}}}: "Biên chế là nền tảng nhân sự để Chi cục thực hiện các nhiệm vụ được giao. Số lượng biên chế được xác định dựa trên Đề án vị trí việc làm và đề nghị của Giám đốc Sở Nội vụ.",\\
\textcolor{blue}{\textbf{\textit{"Linh"}}}: "Cuối cùng, trong trường hợp có vướng mắc hoặc cần điều chỉnh quy định, quy trình giải quyết sẽ được thực hiện như thế nào?",\\
\textcolor{red}{\textbf{\textit{"Minh"}}}: "Chi cục trưởng Chi cục Trồng trọt và Bảo vệ thực vật sẽ báo cáo Giám đốc Sở Nông nghiệp và Phát triển nông thôn để phối hợp với Giám đốc Sở Nội vụ trình UBND tỉnh xem xét và quyết định.",\\
\textcolor{blue}{\textbf{\textit{"Linh"}}}: "Cảm ơn anh Minh đã cung cấp những thông tin quý giá về tổ chức và hoạt động của Chi cục Trồng trọt và Bảo vệ thực vật. Hy vọng qua cuộc trò chuyện này, mọi người sẽ có cái nhìn rõ ràng hơn về vai trò và chức năng của đơn vị.",\\
\textcolor{red}{\textbf{\textit{"Minh"}}}: "Cảm ơn Linh đã tạo cơ hội để tôi chia sẻ. Hy vọng thông tin hữu ích cho mọi người."
\end{minipage}%
}

\section{Script Generated by Llama3.3 Q4}
Below is script generated by Llama3.3-Q4, using the same prompt, document as above. See full output of 3 steps \href{google.com}{here}.


\noindent\fbox{
\begin{minipage}{\textwidth}
\textcolor{blue}{\textbf{\textit{"Linh"}}}: "Xin chào tất cả mọi người, hôm nay chúng ta có sự tham gia của anh Minh để thảo luận về Chi cục Trồng trọt và Bảo vệ thực vật. Anh Minh, trước hết, hãy cho chúng tôi biết chức năng chính của cơ quan này là gì?",\\
\textcolor{red}{\textbf{\textit{"Minh"}}}: "Chức năng chính của Chi cục Trồng trọt và Bảo vệ thực vật là tham mưu, giúp Giám đốc Sở Nông nghiệp và Phát triển nông thôn quản lý nhà nước về lĩnh vực trồng trọt và bảo vệ thực vật trên địa bàn tỉnh.",\\
\textcolor{blue}{\textbf{\textit{"Linh"}}}: "Cảm ơn anh Minh. Vậy, các nhiệm vụ chính của Chi cục bao gồm những gì?",\\
\textcolor{red}{\textbf{\textit{"Minh"}}}: "Các nhiệm vụ chính bao gồm quản lý và hướng dẫn sản xuất trồng trọt, quản lý và bảo vệ thực vật, kiểm tra, thanh tra và xử lý vi phạm trong lĩnh vực trồng trọt và bảo vệ thực vật, thực hiện hợp tác quốc tế, và quản lý tổ chức, biên chế, tài chính, tài sản.",\\
\textcolor{blue}{\textbf{\textit{"Linh"}}}: "Được rồi, vậy cơ cấu tổ chức của Chi cục như thế nào?",\\
\textcolor{red}{\textbf{\textit{"Minh"}}}: "Cơ cấu tổ chức bao gồm Lãnh đạo Chi cục (Chi cục trưởng và Phó Chi cục trưởng) và các phòng chuyên môn, nghiệp vụ như Phòng Hành chính, tổng hợp; Phòng Trồng trọt; Phòng Bảo vệ và Kiểm dịch thực vật.",\\
\textcolor{blue}{\textbf{\textit{"Linh"}}}: "Anh có thể cho biết thêm về việc quản lý biên chế của Chi cục không?",\\
\textcolor{red}{\textbf{\textit{"Minh"}}}: "Biên chế của Chi cục nằm trong tổng biên chế của Sở Nông nghiệp và Phát triển nông thôn được Ủy ban nhân dân tỉnh giao.",\\
\textcolor{blue}{\textbf{\textit{"Linh"}}}: "Chi cục trưởng có những trách nhiệm gì then chốt?",\\
\textcolor{red}{\textbf{\textit{"Minh"}}}: "Chi cục trưởng có trách nhiệm ban hành quyết định về chức năng, nhiệm vụ, quyền hạn cụ thể của các phòng thuộc Chi cục và ban hành Quy chế làm việc của Chi cục cũng như nội quy cơ quan.",\\
\textcolor{blue}{\textbf{\textit{"Linh"}}}: "Tại sao việc quản lý và bảo vệ thực vật lại là một phần quan trọng trong nhiệm vụ của Chi cục?",\\
\textcolor{red}{\textbf{\textit{"Minh"}}}: "Việc quản lý và bảo vệ thực vật là quan trọng vì nó giúp đảm bảo sức khỏe cho cây trồng, giảm thiểu thiệt hại từ sâu bệnh và tăng cường năng suất sản xuất.",\\
\textcolor{blue}{\textbf{\textit{"Linh"}}}: "Vậy làm thế nào để Chi cục có thể thực hiện hiệu quả nhiệm vụ kiểm tra, thanh tra và xử lý vi phạm trong lĩnh vực trồng trọt và bảo vệ thực vật?",\\
\textcolor{red}{\textbf{\textit{"Minh"}}}: "Để thực hiện hiệu quả, Chi cục cần xây dựng và triển khai các kế hoạch kiểm tra, thanh tra thường xuyên, phối hợp với các cơ quan chức năng khác và áp dụng công nghệ để theo dõi và giám sát.",\\
\textcolor{blue}{\textbf{\textit{"Linh"}}}: "Hợp tác quốc tế đóng vai trò gì trong hoạt động của Chi cục?",\\
\textcolor{red}{\textbf{\textit{"Minh"}}}: "Hợp tác quốc tế giúp Chi cục cập nhật kiến thức, công nghệ mới về trồng trọt và bảo vệ thực vật, cũng như học hỏi từ kinh nghiệm của các nước khác để áp dụng vào thực tiễn tại địa phương.",\\
\textcolor{blue}{\textbf{\textit{"Linh"}}}: "Cảm ơn anh Minh vì đã chia sẻ thông tin hữu ích về Chi cục Trồng trọt và Bảo vệ thực vật. Hy vọng rằng, qua cuộc trò chuyện này, chúng ta đã có một cái nhìn tổng quan hơn về vai trò và hoạt động của cơ quan này trong lĩnh vực nông nghiệp."\\  
\end{minipage}
}

\end{document}
